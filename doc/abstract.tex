\chapter*{Abstract}
\addcontentsline{toc}{chapter}{Abstract}
Computer systems are used to store and protect highly sensitive data. At the
same time the resources and determination of attackers have increased
significantly. Therefore, systems capable of safeguarding critical information
must conform to rigorous quality standards to establish trust in the correct
functioning.

Attaining high assurance for today's monolithic operating systems is
exceptionally hard since they tend to be large and include complex
functionality. This fact is highlighted by regular security updates provided by
operating system vendors. Thus they are a weak foundation for building secure
systems.

In contrast, microkernels can be well suited as a basis for systems with strict
demands on robustness. They are by definition small, which is a precondition
for rigorous verification of correctness. Additionally, they lend themselves to
the construction of component-based systems where the incorrect behavior of one
partition does not impact the whole system.

A separation kernel (SK) is a specialized microkernel that provides an
execution environment for multiple components that can only communicate
according to a given policy and are otherwise isolated from each other. Hence,
the isolation also includes the limitation of potential side- and covert
channels. SKs are generally more static and smaller than dynamic
microkernels, which minimizes the possibility of kernel failure and should
simplify the application of formal verification techniques.

Recent addition of advanced hardware virtualization support for the Intel x86
architecture has the potential of greatly simplifying the implementation of a
separation kernel which can support complex systems.

This thesis presents a design of a separation kernel for the Intel x86
architecture using the latest Intel hardware features. An open-source prototype
written in SPARK demonstrates the viability of the envisioned concept and the
application of SPARK's proof capabilities increases the assurance of the
correctness of the implementation.
