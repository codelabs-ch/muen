\chapter{Introduction}
TODO

\section{Notation}
This section presents the notational conventions used throughout this document.

\subsection{Keywords}
Important terms and concepts that are introduced for the first time are
presented in an italic style, e.g. \emph{subject specification}. Subsequent
occurences of the same term have no special formatting.

\subsection{Numbers}
Regular numbers that have no leading special character are expressed as decimal
values. Hexadecimal numbers such as memory addresses are explicitly preceded by
\texttt{0x}.

Storage units such as kilo-, mega- and gigabyte are designated by the common
abbreviations KB, MB and GB.

\section{Related literature}
Since the target hardware platform of the separation kernel is the Intel x86
architecture, its specification called "Intel\textsuperscript{\textregistered}
64 and IA-32 Architectures Software Developer's Manual" \cite{IntelSDM} is the
main source of technical information about the hardware platform. The documents
are commonly referred by their short name Intel SDM.

The books are available online and updated by Intel on a regular basis. This
can lead to changes in the document structure. The chapter and section
citations given in this report refer to the Intel SDM revision 44, released in
August 2012.
