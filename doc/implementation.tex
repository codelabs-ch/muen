\chapter{Implementation}
\section{Policy}
All aspects of a system using the Muen kernel must be specified in a policy XML
file. The policy is composed of the following main parts:

\begin{itemize}
	\item Hardware
	\item Kernel
	\item Binaries
	\item Subjects
	\item Scheduling
\end{itemize}

XML was chosen as a specification language since it is human-readable and can
be automatically verified against a schema. Furthermore, there is an existing
Ada library called XML/Ada TODO:Ref, which is opensource and freely available.

Since XML processing is not done by any trusted part of the system, the policy
contained in an XML file is transformed into SPARK source using the policy
compilation tool \texttt{skpolicy}. This process is described in detail in
section TODO:Ref.

Each of the main policy parts is presented in the following sections. Preceding
these descriptions is the specification of data types, which are the basis for
the subsequent definition of policy elements. The data types and elements map
directly to their corresponding XML schema definitions.

\subsection{Data types}
This section describes basic data types that are used in the specification of
the system policy. They are referenced in later chapters, illustrating different
parts of the policy.

\input{types.tex}

\subsection{Hardware}
\label{subsec:hardware}
\begin{figure}[h]
	\centering
	\includegraphics[width=\textwidth]{images/xml_hardware.png}
	\caption{Hardware policy}
\end{figure}
\input{hardware.tex}

\subsection{Kernel}
\begin{figure}[h]
	\centering
	\includegraphics[scale=0.6]{images/xml_kernel.png}
	\caption{Kernel policy}
\end{figure}
\input{kernel.tex}

\subsection{Binaries}
\begin{figure}[h]
	\centering
	\includegraphics[scale=0.6]{images/xml_binary.png}
	\caption{Binaries policy}
\end{figure}
\input{binary.tex}

\subsection{Subjects}
\begin{sidewaysfigure}[hp]
	\includegraphics[width=\textwidth]{images/xml_subject.png}
	\caption{Subjects policy}
\end{sidewaysfigure}
\input{subject.tex}

\subsection{Scheduling}
\begin{figure}[h]
	\centering
	\includegraphics[scale=0.6]{images/xml_scheduling.png}
	\caption{Scheduling policy}
\end{figure}
\input{scheduling.tex}

\section{Zero Footprint Runtime}\label{sec:zfp-rts}
To allow a small TCB for the kernel and subjects, a special stripped-down
version of an Ada runtime\index{runtime} is provided by the Muen project. This
runtime system (RTS\index{RTS}) contains the minimal number of packages
required to compile Ada code for Muen. An Ada runtime system with such a lean
"footprint" is called \emph{Zero Footprint Runtime} (ZFP\index{ZFP}). The
following is a list of currently supported functions and packages:

\begin{itemize}
	\item function \texttt{Ada.Unchecked\_Conversion}
	\item package \texttt{Interfaces}
	\item package \texttt{System}
	\item package \texttt{System.Machine\_Code}
	\item package \texttt{System.Storage\_Elements}
\end{itemize}

These files have been extracted from the latest sources of the GNU Compiler
Collection (GCC\index{GCC}) \cite{gcc} and are linked into a static library.
The runtime is then used to compile the Muen kernel and all native subjects
running on the kernel.

Since the ZFP runtime provides only a very limited set of Ada language features,
code using this runtime must be also very simple. Simple code is an important
pre-condition for formal verification.

The compliance of the Muen kernel to the SPARK\index{SPARK} language rules
already assures that no complex Ada features are used. Additional restriction
pragmas\index{pragma} confine the language subset even further. The currently
used restrictions are shown in listing \ref{lst:pragmas}.

\lstinputlisting[
	language=Ada,
	label=lst:pragmas,
	caption=Restriction pragmas]
	{../../config/restrictions.adc}

Details about restriction pragmas and their impact on the allowed Ada language
subset can be found in the GNAT Reference Manual, section 4 "Standard and
Implementation Defined Restrictions" \cite{GNAT:manual}.

\section{Subject}
Subjects are the main components that are executed on top of the kernel, as
described in section TODO:Ref. They are represented using two main data
structures: \emph{subject specification} and \emph{subject state}.

\subsection{Specification}
A subjects is specified in the global system policy. As described in the
previous section TODO:Ref, the XML specification precisely defines the execution
environment and granted resources. The information that is relevant to the
kernel is part of the compiled policy. Listing \ref{lst:skp-subjects} presents
the SPARK type specification into which all relevant subject policy parts are
transformed.

\lstinputlisting[
	language=ada,
	linerange={14-33},
	label=lst:skp-subjects,
	caption=SPARK subject spec type]
	{../tools/policy/templates/skp-subjects.adb}

The specification of a subject is static and cannot be changed at runtime.

\subsection{State}
The system state related to a subject must encompass all resources, that it can
control directly or indirectly. This is necessary to enable the scheduler to
preempt a subject by preserving its state and seamlessly resume execution at a
later stage by restoring the previously saved state.
Additionally, separation of subjects demands that unintended information flows
when switching between subjects must be prevented. This is only achievable if
the subject state that is saved and restored encompasses every element of the
systems environment that is accessible by more than one subject. Listing
\ref{lst:sk-subject-state} shows the record type used by ther Kernel to maintain
the state of a subject.

\lstinputlisting[
	language=ada,
	linerange={40-56},
	label=lst:sk-subject-state,
	caption=SPARK subject state type]
	{../common/src/sk.ads}

A special monitoring subject called \emph{subject monitor}
\index{subject monitor} (SM) may be given access to parts or the entire state of
a given subject. This enables the SM to alter a subject's state given the right
conditions and thus perform emulation.


\section{Kernel}
\subsection{Init}\label{subsec:init}
After reset of a x86 system, the processor begins executing code at physical
address \texttt{ffff:0000}, which is mapped to the PC
BIOS\index{BIOS}\footnote{Basic Input/Output System}. The BIOS first performs
tests and initialization routines and then searches for a bootable storage
media. If found, the BIOS copies the first sector of the storage media to
physical address \texttt{0000:7C00} and jumps to this address (i.e. starts
executing code at this address). This is where the system bootloader comes to
live which is responsible to boot operating systems according to its
configuration. Many bootloaders first load additional code from the storage
media and then prepare the environment for OS execution.

The Muen seperation kernel is compliant to the multiboot specification, version
0.6.96 \cite{multiboot}. The multiboot standard is used to uniformly boot
different operating system kernels by multiboot-aware bootloaders.
The Muen kernel exports the required multiboot header within the first 8192
bytes of the OS image. The bootloader loads the OS image into memory according
to the information found in the header and jumps to the physical kernel entry
point specified in this header.

It is the bootloader's task to prepare the system state as demanded by the
multiboot standard, see \cite{multiboot} section 3.2 for details. The system
kernel can except the system to be in this state. After the Muen kernel comes to
live, it performs additional steps before jumping into the main SPARK kernel.
This initial startup code is written in Assembly and conducts the following
tasks:
\begin{enumerate}
	\item Copy the AP trampoline to low-memory, see section
		\ref{subsec:mp-support} \item Initialize per-CPU VMXON regions
	\item Initialize subject VMCS regions
	\item Enable PAE\index{PAE}\footnote{Physical Address Extension}
	\item Initialize per-CPU kernel pagetables
	\item Enable IA-32e mode and execute-disable (NX)
	\item Enable paging, write protection, caching and native FPU error
		reporting
	\item Set up 64-bit GDT\index{GDT}\footnote{Global Descriptor Table}
	\item Set up Page-Attribute Table (PAT)
	\item Set up kernel stack
	\item Initialize Ada runtime
	\item Jump into kernel main
\end{enumerate}

The system is now in 64-bit IA-32e mode and each kernel calls the \texttt{VMXON}
VT-x instruction to enter VMX root operation.

\subsection{Scheduling}\label{subsec:scheduling}
After system initialization is complete, the kernels running on the different
logical processors synchronize and start scheduling subjects. Which subject to
schedule on what particular CPU is defined by the scheduling plan. A policy
writer defines the system's scheduling regime in the XML policy file, see
listing \ref{lst:xml-scheduling-plan} for an example.

\begin{lstlisting}[
	language=xml,
	label=lst:xml-scheduling-plan,
	caption=System scheduling plan in XML]
<scheduling tick_rate="10000">
	<major_frame>
		<cpu>
			<minor_frame subject_id="1" ticks="40"/>
			<minor_frame subject_id="2" ticks="40"/>
		</cpu>
		<cpu>
			<minor_frame subject_id="3" ticks="80"/>
		</cpu>
	</major_frame>
	<major_frame>
		<cpu>
			<minor_frame subject_id="1" ticks="80"/>
		</cpu>
		<cpu>
			<minor_frame subject_id="4" ticks="80"/>
		</cpu>
	</major_frame>
</scheduling>
\end{lstlisting}

The example scheduling plan of listing \ref{lst:xml-scheduling-plan} contains
four subjects which are scheduled on two logical processors. The Muen policy
compiler transforms the scheduling plan in XML format to SPARK specifications
which are directly compiled into the kernel. The scheduling plan is therefore
static at compilation time and cannot be modified at runtime.

\begin{figure}[h]
	\centering
	\begin{tikzpicture}[minimum height=0.6cm]
	\node (sch) [bluebox]                {Scheduler};
	\node (knl) [bluebox, left=of sch]   {Kernel Main};
	\node (pln) [apribox, above=of sch]  {Scheduling Plan};
	\node (sub) [greenbox, right=of sch] {Subject};
	\node[gray, font=\scriptsize] at (0.8,2.3) {VMX root};
	\node[gray, font=\scriptsize] at (2.6,2.3) {VMX non-root};

	\draw[arrow] (knl) to (sch);
	\draw[arrow] (pln) to (sch);
	\draw[arrow] (sch) to[bend right=65] node[auto] {VM enter} (sub);
	\draw[arrow] (sub) to[bend right=65] node[auto] {VM exit}  (sch);
	\draw[thin, dotted, gray] (1.6,-1.5) to (1.6,2.5);
\end{tikzpicture}

	\caption{Kernel scheduler}
	\label{fig:kernel-scheduler}
\end{figure}

Each kernel stores the index of the active minor frame in its per-CPU storage
area. The index points into the current scheduling major frame. Initially, this
value is set to one (\texttt{Minor\_Frame\_Range'First}), i.e. the first minor
frame in the active major frame.

The index designating the current major frame is global and therefore identical
for all kernels. This index is managed by the privileged $\tau$0 subject and
only read by the kernels.

The minor/major frame tuple forms an index into the scheduling plan. It points
to a minor frame containing the subject ID and timer ticks for the next subject
to schedule. The subject ID is used to load the state of the corresponding
subject from the state descriptor array into the processor.

The mechanism used to keep track of time is the VMX preemption timer. Writing
the timer ticks into the subject's VMCS region sets the timer. The kernel then
calls the \texttt{VMLAUNCH} or \texttt{VMRESUME} (if the subject has already
been launched) VT-x instructions to enter VMX non-root operation, which lets the
processor execute subject code.  The subject is then automatically preempted by
the processor when the allotted time slice is over. Figure
\ref{fig:kernel-scheduler} illustrates this process.

The subject state is saved on each VM exit. Depending on the scheduling plan and
the exit reason, the state of another subject is loaded by the kernel scheduler.
If the exit occurred because the VMX preemption timer fired, the scheduler is
aware that it must advance to the next minor frame in the current major frame.
This is done by incrementing the current minor frame counter. If the minor frame
index reaches the upper end of the allowed range
(\texttt{Minor\_Frame\_Range'Last}), it is reset to the first value of the
range.

\subsection{Traps}\label{subsec:traps}
Transitions from VMX non-root operation to VMX root operation are called VM
exits (\cite{IntelSDM}, volume 3c section 23.3). Because processor behavior in
VMX non-root mode is limited, reasons for VM exits can be the execution of a
privileged operation or an instruction that has been constrained by the
appropriate VMX controls.

The Muen kernel uses the term \emph{trap} to handle a VM exit. The system policy
allows the specification of a per-subject trap table, which defines what action
to take when a trap occurs. Listing \ref{lst:trap-table} shows an example trap
table.

\begin{lstlisting}[language=xml, label=lst:trap-table, caption=Subject trap table]
<trap_table>
    <entry kind="*" dst_subject="sm" dst_vector="36"/>
</trap_table>
\end{lstlisting}


This single trap table entry defines that all configurable traps should result
in an execution handover to a subject called \emph{sm} (Subject
Monitor\index{SM}). Additionally, an interrupt vector 36 should be injected into
the handler subject on handover. Because a trap handler subject performs
operations in place of the causing subject, a trap always results in a handover
(i.e. the trapping subject is removed from the scheduling plan and replaced by
the destination subject).

Valid trap kinds are the VMX basic exit reasons defined by Intel in
\cite{IntelSDM}, volume 3C appendix C. Four exit reasons or trap kinds are
excluded from the list of configurable traps because they are reserved for
internal use by the Muen kernel.

\begin{itemize}
	\item \emph{External interrupt} (reason 1)\\
		The external interrupt trap is used to implement external interrupt
		delivery to subjects as explained in section \ref{subsec:external-ints}.
	\item \emph{Interrupt window} (reason 7)\\
		The interrupt window trap is used by the Muen kernel to optimize the
		latency of interrupt injection.
	\item \emph{VMCALL} (reason 18)\\
		Used in the event mechanism to provide the hypercall interface, see
		section \ref{subsec:events} for more details.
	\item \emph{VMX-preemption timer expired} (reason 52)\\
		Used by the kernel scheduler to preempt subjects
		(\ref{subsec:scheduling}).
\end{itemize}

If one of the reserved traps occurs, the kernel invokes the appropriate handler
procedure. All other trap kinds can be used to configure subject trap table
entries. When a configured trap occurs, the kernel consults the static trap
table of the subject to check its validity. If it is ok, a handover is performed
to the destination subject as defined by the trap table entry. Listing
\ref{lst:trap-table-spec} shows an example trap table specification generated by
the \texttt{skpolicy} tool.

\begin{lstlisting}[language=Ada, label=lst:trap-table-spec, caption=Trap table specification]
Trap_Table => Trap_Table_Type'(
  0      => Trap_Entry_Type'(Dst_Subject => 2, Dst_Vector => 256),
  48     => Trap_Entry_Type'(Dst_Subject => 2, Dst_Vector => 12),
  others => Null_Trap),
\end{lstlisting}

Subject with ID two is used as handler for trap kinds "exception or
NMI\index{NMI}" (0) and "EPT violation" (48). All other traps are invalid for
his subject.

The trap mechanism is most commonly used to implement the "trap and emulate"
mechanism: A subject executes a privileged operation resulting in a trap. The
subject's trap table defines which handler subject is responsible for processing
the trap. The handover is performed by the kernel and the trap handler subject
emulates the privileged operation by directly modifying the trapping subject's
memory or architectural state. After the operation is complete, the handler
subject resumes execution of the subject which caused the trap by using
a handover event as described in section \ref{subsec:events}.


\subsection{Exceptions and Interrupts}\label{subsec:excp-and-ints}
\begin{figure}[h]
	\centering
	\begin{tikzpicture}
	\node[graybox] (han) {\textbf{3} Handle IRQ};
	\node[above=2mm of han] (mue) {Muen SK};

	\begin{pgfonlayer}{background}
		\node[bluebox, minimum width=3cm, minimum height=1.7cm] (mub) [fit = (han) (mue)] {};
	\end{pgfonlayer}

	\node[greenbox, minimum width=3cm, minimum height=1.7cm, above=of mub] (sub) {Subject};
	\node[apribox, left=15mm of mub] (irq) {Keyboard};

	\draw[arrow] (irq) to node[auto] {\textbf{1} IRQ 1} (mub);
	\draw[arrow] (sub.225) to node[auto, swap] {\textbf{2} VM exit} (mub.135);
	\draw[arrow] (mub.45) to node[auto, swap] {\textbf{4} Inject event} (sub.315);
\end{tikzpicture}

	\caption{External interrupt handling}
	\label{fig:external-interrupt}
\end{figure}

\subsection{Multicore support}\label{subsec:mp-support}
The Muen separation kernel makes use of all logical processors available in a
system. The processor count of a specific hardware platform is specified in the
system policy, see section \ref{subsec:hardware}. This section describes how the
multicore setup is done on kernel startup.

Modern PC systems comply to the Intel MultiProcessor (MP\index{MP})
specification. In short, the Intel MP specification is an open-standard
describing enhancements to both operating systems and firmware to be able to
init, boot and operate x86 multiprocessor systems. For more information see
\cite{intel:mp}.

After the hardware completed its part of the MP specification, one processor
has been negotiated to be the bootstrap processor (BSP\index{BSP}). All other
logical processors, called application processors (AP\index{AP}), halt until
they receive a specific inter-processor interrupt (IPI\index{IPI}) sequence.

\begin{figure}[h]
	\centering
	\begin{tikzpicture}
	\node[greenbox, minimum width=9cm] (mem) {System Memory};

	% SK 0
	\node[graybox, minimum width=2.8cm, below=1cm of mem.south west, anchor=north west] (pc1) {Per-CPU storage};
	\node[graybox, minimum width=2.8cm, below=1mm of pc1] (st1) {Stack};
	\node[above=2mm of pc1] (mu1) {Muen SK};
	\begin{pgfonlayer}{background}
		\node[bluebox, minimum width=3cm, minimum height=1.7cm] (mb1) [fit = (pc1) (st1) (mu1)] {};
	\end{pgfonlayer}

	\node[apribox, minimum width=3cm, below=5mm of mb1, label=below:\emph{BSP}] (cp1) {CPU0};

	\draw[arrow, gray] (mb1) to node[auto, gray] {LAPIC} (cp1);

	% SK 1
	\node[graybox, minimum width=2.8cm, below=1cm of mem.south east, anchor=north east] (pc2) {Per-CPU storage};
	\node[graybox, minimum width=2.8cm, below=1mm of pc2] (st2) {Stack};
	\node[above=2mm of pc2] (mu2) {Muen SK};
	\begin{pgfonlayer}{background}
		\node[bluebox, minimum width=3cm, minimum height=1.7cm] (mb2) [fit = (pc2) (st2) (mu2)] {};
	\end{pgfonlayer}

	\node[apribox, minimum width=3cm, below=5mm of mb2, label=below:\emph{AP}] (cp2) {CPU1};

	\draw[arrow, gray] (cp2) to node[auto, gray] {LAPIC} (mb2);

	% Inter-core
	\draw[arrow, gray] (mb1) to node[gray, auto] {INIT-SIPI-SIPI} (mb2);
\end{tikzpicture}

	\caption{Multicore architecture}
	\label{fig:mp-overview}
\end{figure}

The BSP starts executing code as describe in section \ref{subsec:init}. The
init code initializes the system and jumps into the main SPARK kernel. The
kernel running on the BSP is responsible to bootstrap the other application
processors. It first enables its local APIC\index{APIC} to be able to send
inter-processor interrupts to the halted AP processors. To wakeup the APs, the
INIT-SIPI-SIPI IPI sequence must be sent to their APICs, as described in the MP
specification. See also figure \ref{fig:mp-overview} for an illustration of
this process. The SIPI IPI contains the physical address vector of the
trampoline code copied to low-memory by the init code. The AP processors jump
to this code after wakeup. The trampoline performs the following steps:

\begin{enumerate}
	\item Set up 32-bit GDT\index{GDT}
	\item Switch CPU to protected mode
	\item Initialize DS and SS segments
	\item Jump to the AP entry point in the init code
\end{enumerate}

These steps initialize the APs to the same architectural state as the
bootloader did for the BSP: 32-bit protected mode with paging disabled.
Therefore, the final step is to let the AP processors jump to the identical
init code described in section \ref{subsec:init}.

\subsubsection{Per-kernel memory}
The Muen kernels operate fully symmetrical, i.e. code running on the different
logical processors is (binary) identical. Nevertheless, each kernel owns a
distinct stack page and also a page to store per-CPU data. This however is fully
transparent to the kernels as their virtual stack and global storage addresse
values are the same. This is achieved by using different page table structures
for each kernel. Page tables are created by the policy tool and setup on system
startup by the init code. The main kernel has no access to these structures in
memory and does not bother with memory management.

\subsubsection{Synchronization} Since synchronization is error-prone and it is
desirable to reduce inter-core dependencies as much as possible, the Muen
kernel tries to avoid locks and other synchronization primitives. Nevertheless
minimal synchronization is required at certain key points in the code. This
section describes the spinlock and barrier mechanisms used by the kernel.

\paragraph{Spinlock}
The spinlock implementation uses the \texttt{XCHG} processor instruction to
atomically swap the value one with the contents of a lock variable in memory.
If the result of the set operation is zero, no other core currently holds the
lock and it is successfully acquired. If the result is one, the lock is
currently busy and the core must spin and retry again.

Inside the lock's busy loop the \texttt{PAUSE} instruction is used to improve
performance and resource utilization on CPUs with hyper-threading
(HTT\index{HTT}) enabled TODO:REF.

\paragraph{Barrier}
As described in section \ref{subsec:scheduling}, to guarantee temporal
separation, the scheduling plans on the different logical processors must be
synchronized on major frame transition.

This is achieved by ...

\subsection{Events}\label{subsec:events}
The event\index{event} mechanism provided by the Muen kernel is used for
inter-subject signalisation. A subject is allowed to send an event to another
subject if this operation has been granted by an entry in the subject's policy
event table. The following listing is used as an example to illustrate the
event mechanism.

\begin{lstlisting}[language=xml, label=lst:event-table, caption=Subject event table]
<event_table>
    <interrupt event="1" dst_subject="s2" dst_vector="33" send_ipi="true"/>
    <handover  event="2" dst_subject="s3"/>
</event_table>
\end{lstlisting}


This event table in the system policy allows the associated subject to send two
events of different type to subjects \emph{s2} and \emph{s3} respectively. A
handover event transfers execution to a destination subject, optionally
injecting an interrupt.  Interrupt events inject an interrupt in a destination
subject, emitting an optional inter-processor interrupt (IPI)\index{IPI} to
speed up inter-core interrupt delivery.

Interrupts are injected into the destination subject by the Muen kernel to
inform it about new pending events. In this example, interrupt vector 33 is
injected into subject s2 for the interrupt event and no interrupt is injected
for the handover event.

If the IPI option is enabled for an interrupt event, the Muen kernel sends an
inter-processor interrupt to the logical CPU of the destination subject. The
IPI causes the target processor to trap into the kernel, which results in the
preemption of the running subject and therefore the immediate delivery of the
interrupt on next subject entry. The event IPI option is only valid if the
destination subject runs on a different logical CPU than the sending subject.
This is enforced by the policy compiler.

To signal an event, subjects implemented in Ada or SPARK can use the
\texttt{SK.Hypercall} package provided by Muen. The package contains the
procedure \texttt{Trigger\_Event} which accepts the event number as its sole
argument. The procedure wraps the \texttt{VMCALL} VMX instruction, which causes
a trap into the kernel when called from VMX non-root mode. This is used by the
Muen kernel to implement the hypercall mechanism.  With the Intel VMX
extensions, a hypercall is a special trap with basic exit reason number 18.

The Muen kernel handles hypercall traps separately in the
\texttt{Handle\_Hypercall} procedure. It performs a lookup in the sending
subject's event table to find the associated event entry. If no such entry
exists, the kernel logs an error message (in debug mode only) and ignores the
event. Valid events of type interrupt are injected using the VMX interrupt
injection capabilities.

If the event is of type handover, the sending subject is replaced by the
destination subject in the scheduling plan. The destination subject is then
executed in place of the sending subject until it yields the CPU in favor of
another subject. This method can be used to implement co-operative scheduling
in subject groups.

\begin{figure}[h]
	\centering
	\begin{tikzpicture}
	% SK 0
	\node[graybox] (ha1) {Handle Hypercall};
	\node[above=2mm of ha1] (mu1) {Muen SK};
	\begin{pgfonlayer}{background}
		\node[bluebox, minimum width=3cm, minimum height=1.7cm] (mb1) [fit = (ha1) (mu1)] {};
	\end{pgfonlayer}

	\node[apribox, minimum width=3cm, below=5mm of mb1] (cp1) {CPU0};
	\node[greenbox, minimum width=3cm, minimum height=1.7cm, above=of mb1] (su1) {Subject};

	\draw[arrow] (su1.225) to node[auto, swap] {\textbf{1}} (mb1.135);
	\draw[arrow] (mb1.45) to node[auto, swap] {\textbf{4}} (su1.315);
	\draw[<->, thick] (cp1) to node[auto] {LAPIC} (mb1);

	% SK 1
	\node[graybox, right=2.6cm of ha1] (ha2) {Handle Hypercall};
	\node[above=2mm of ha2] (mu2) {Muen SK};
	\begin{pgfonlayer}{background}
		\node[bluebox, minimum width=3cm, minimum height=1.7cm] (mb2) [fit = (ha2) (mu2)] {};
	\end{pgfonlayer}

	\node[apribox, minimum width=3cm, below=5mm of mb2] (cp2) {CPU1};
	\node[greenbox, minimum width=3cm, minimum height=1.7cm, above=of mb2] (su2) {Subject};

	\draw[arrow] (mb2.135) to node[auto, swap] {\textbf{2}} (su2.225);
	\draw[arrow] (su2.315) to node[auto, swap] {\textbf{3}} (mb2.45);
	\draw[<->, thick] (cp2) to node[auto] {LAPIC} (mb2);

	% Inter-core
	\draw[<->, thick] (mb1) to node[auto] {IPI} (mb2);
	\draw[vecarrow] (su1.15) to node[auto] {Request page} (su2.165);
	\draw[vecarrow] (su2.195) to node[auto] {Response page} (su1.345);
\end{tikzpicture}

	\caption{Inter-core events}
	\label{fig:inter-core-events}
\end{figure}

The event mechanism in general can be utilized to implement low-latency
communication channels between subjects, as illustrated in figure
\ref{fig:inter-core-events}. In this example, two subjects running on different
logical CPUs (CPU0, CPU1) implement a simple request-response design:

\begin{enumerate}
	\item The requesting subject on the left (client) running on CPU0 writes
		the request data into a memory page shared with the service provider
		subject on the right (server). It then signals a pending request to the
		server subject by sending an event (i.e. it calls the
		\texttt{Trigger\_Event} procedure).

		The event lets the processor trap into the kernel. Since the subjects
		run on different cores in parallel, interrupt events must be used in
		this example. The \texttt{Handle\_Hypercall} procedure in the kernel
		checks if the event is valid and inserts it into the destination
		subject's pending event list. If the IPI option is enabled, the kernel
		also sends an IPI to the target CPU1.
	\item The IPI causes a trap on the destination CPU1. The interrupt
		event is therefore immediately injected into the destination subject on
		the next VMX entry.

		The server subject is waiting for data so it might be in a halted
		state.  The injected interrupt resumes the server subject which then
		reads the request data from the (read-only) request page and calculates
		the result. It writes the result data into the response page shared
		with the client subject.
	\item The server subject signals completion by triggering an event, which
		again leads to a trap into the kernel. The kernel inserts the event in
		the pending event list of the client subject and (if enabled) sends an
		IPI to CPU0.
	\item The pending event is injected into the client subject. The subject
		might be in a halted state because it is waiting for the result. The
		injected interrupt resumes the subject and the service result can be
		copied from the response page shared with the server.
\end{enumerate}

\section{Build}
This section outlines the different build steps required to produce the final
system image containing the Muen separation kernel and all subjects as defined
by the system policy. Figure \ref{fig:build-process} illustrates the build
process.

\begin{figure}[h]
	\centering
	\begin{tikzpicture}[minimum height=0.6cm]
	\node (src) [apribox]               {Source Format};
	\node (mrg) [bluebox, right=of src] {Merger};
	\node (exp) [bluebox, right=of mrg] {Expander};
	\node (fma) [apribox, right=of exp] {Format A};
	\node (alo) [bluebox, right=of fma] {Allocator};
	\node (fmb) [apribox, below=of alo] {Format B};
	\node (val) [bluebox, left=of fmb]  {Validator};
	\node (sge) [bluebox, left=of val]  {Structure Generators};

	\node (spk) [graybox, below=of sge] {Source Specs};
	\node (iob) [graybox, right=of spk] {I/O Bitmaps};
	\node (msc) [graybox, right=of iob, minimum width=1.5cm] {...};
	\node (pta) [graybox, left=of spk] {Page Tables};
	\node (zpa) [graybox, left=of pta] {Zero Page};

	\node (bin) [apribox, below=of spk] {Binaries};

	\node (has) [bluebox, below=of bin] {Hasher};

	\node (sin) [bluebox, below=of has] {Sinfo};

	\node (pak) [bluebox, below=of sin] {Packer};

	\node (img) [apribox, below=of pak] {System Image};

	\draw[arrow] (src) -- (mrg);
	\draw[arrow] (mrg) -- (exp);
	\draw[arrow] (exp) -- (fma);
	\draw[arrow] (fma) -- (alo);
	\draw[arrow] (alo) -- (fmb);
	\draw[arrow] (fmb) -- (val);
	\draw[arrow] (val) -- (sge);

	\draw[arrow] (sge) -- (spk);
	\draw[arrow] (sge) -- (iob);
	\draw[arrow] (sge) -- (msc);
	\draw[arrow] (sge) -- (pta);
	\draw[arrow] (sge) -- (zpa);

	\draw[arrow] (spk) -- (bin);

	\draw[arrow] (bin) -- (has);
	\draw[arrow] (has) -- (sin);
	\draw[arrow] (sin) -- (pak);

	\draw[arrow] (msc) -- (pak);
	\draw[arrow] (iob) -- (pak);
	\draw[arrow] (iob) -- (pak);
	\draw[arrow] (pta) -- (pak);
	\draw[arrow] (zpa) -- (pak);

	\draw[arrow] (pak) -- (img);
\end{tikzpicture}

	\caption{Build process}
	\label{fig:build-process}
\end{figure}

The first step is to build the tools. The \texttt{skconfig} helper tool can be
used, as described in section \ref{subsec:subject-binary-analysis}, to create
subject initial state and memory layout specifications in XML format from the
subject binary. The generated XML files are included in the system policy before
the policy compilation step.

To compile the system policy, the \texttt{skpolicy} tool is required. The
details about the policy compilation process is described in section
\ref{subsec:policy-compilation}.  The Muen kernel and all subjects require the
Ada ZFS runtime described in section \ref{sec:zfs-rts} to compile. Once the
policy has been compiled and the ZFS runtime has been built, the kernel and all
subjects are built.

The final step is to package all object binaries (i.e. the kernel and all
subjects) including all related files into a bootable OS image. This task is
handled by the \texttt{skpacker} tool described in section
\ref{subsec:image-packaging}.

\subsection{Policy compilation}\label{subsec:policy-compilation}
The \texttt{skpolicy} tool compiles the XML system policy outlined in section
\ref{sec:policy} to different output formats as shown in figure
\ref{fig:policy-compilation}.

\begin{figure}[h]
	\centering
	\begin{tikzpicture}
	\node (pol) [greenbox]                            {Policy};
	\node (skp) [apribox, below=of pol]               {skpolicy};
	\node (sou) [bluebox, below=of skp, xshift=1.5cm] {Source specs};
	\node (pag) [bluebox, left=of sou]                {Page tables};
	\node (iob) [bluebox, left=of pag]                {I/O bitmaps};
	\node (msr) [bluebox, right=of sou]               {MSR bitmaps};

	\draw[arrow] (pol) to (skp);
	\draw[arrow] (skp) to (sou);
	\draw[arrow] (skp) to (msr);
	\draw[arrow] (skp) to (pag);
	\draw[arrow] (skp) to (iob);
\end{tikzpicture}

	\caption{Policy compilation}
	\label{fig:policy-compilation}
\end{figure}

The generated subject page tables, I/O and MSR bitmaps are included in the final
system image by the \texttt{skpacker} utility explained in the following section
\ref{subsec:image-packaging}. These files are all in binary form and correspond
to the format mandated by the respective Intel SDM chapters \cite{IntelSDM}.
Page tables are generated for subjects as well as for the kernel itself.

The generated SPARK and Assembly source specifications are included in the
kernel directly. These specifications provide the kernel with the following
information:

\begin{itemize}
	\item \emph{IRQ routing specification}\\
		Used by the kernel to program the system's I/O APIC for interrupt
		routing.
	\item \emph{Kernel address constants}\\
		Define kernel stack, page table and per-CPU storage memory addresses.
	\item \emph{Scheduling plans for all CPU cores}\\
		The scheduling plans are indexed by the logical processor's APIC ID.
		Each kernel copies its associated scheduling plan to the per-CPU storage
		area on initialization.
	\item \emph{Subjects specification}\\
		Defines all subjects and their parameters, see policy section
		\ref{subsec:subjects}.
	\item \emph{Packer specification}\\
		Defines the configuration used for the \texttt{skpacker} tool outlined
		in the following section \ref{subsec:image-packaging}.
\end{itemize}

\subsection{Subject binary analysis}\label{subsec:subject-binary-analysis}
The \texttt{skconfig} tool uses the Binary File Descriptor (BFD\index{BFG})
library to analyze subject binaries and creates appropriate XML\index{XML}
policy specifications from it, see figure \ref{fig:object-analysis}. This is
useful to generate the initial state and the memory layout directly from a
native subject binary instead of writing it by hand. The tool extracts stack
address, entry point and memory layout from a subject binary.

\begin{figure}[h]
	\centering
	\begin{tikzpicture}
	\node (obj) [bluebox]                {Binary object};
	\node (skc) [apribox,  below=of obj] {skconfig};
	\node (xml) [greenbox, below=of skc] {XML specification};

	\draw[arrow] (obj) to (skc);
	\draw[arrow] (skc) to (xml);
\end{tikzpicture}

	\caption{From binary object to XML specification}
	\label{fig:object-analysis}
\end{figure}

The exctracted subject initial state and memory layout XML specifications can be
included in the system policy before compilation by the \texttt{skpolicy} tool.
The generated memory layout is only as permissive as required by the original
subject binary. For example, the memory region mapped for executable code
(the \texttt{.text} section) will be executable but non-writeable. This is in
contrast to just providing a big enough memory region with all permissions to
the subject with no exact mapping of binary sections to memory access
permissions (i.e. read, write, execute).

\subsection{Image packaging}\label{subsec:image-packaging}
The \texttt{skpacker} tool is responsible to assemble the final bootable system
image from the parts produced in the previous build steps. Figure
\ref{fig:image-packaging} illustrates the process. The tool includes the special
packer source specification created from the system policy. This specification
includes information about subject binaries and their physical address in the
final image.

\begin{figure}[h]
	\centering
	\begin{tikzpicture}
	\node (knl) [bluebox]                              {Kernel binary};
	\node (sub) [bluebox, left=of knl]                 {Subject binaries};
	\node (pag) [bluebox, left=of sub]                 {Page tables};
	\node (bit) [bluebox, right=of knl]                {Bitmaps};
	\node (skp) [apribox, below=of knl, xshift=-1.5cm] {skpacker};
	\node (spe) [bluebox, left=of skp]                 {Packer spec};
	\node (sys) [greenbox, below=of skp]               {System image};

	\draw[arrow] (pag) to (skp);
	\draw[arrow] (sub) to (skp);
	\draw[arrow] (knl) to (skp);
	\draw[arrow] (bit) to (skp);
	\draw[arrow] (skp) to (sys);
	\draw[arrow] (spe) to (skp);
\end{tikzpicture}

	\caption{System image packaging}
	\label{fig:image-packaging}
\end{figure}

The remaining configuration is extracted from the source specification provided
to the kernel. Listing \ref{lst:skpacker} shows the output of a
\texttt{skpacker} run with the example system described in section
\ref{sec:example-system}. The fist column in the output designate physical
addresses in memory. The second column specifies the type of the packaged file
at this specific memory location. The abbreviations have the following meaning:

\begin{itemize}
	\item \emph{PML4} The file at this address designates a page table
		structure. It is either a page table for a kernel or for a subject. The
		kernels running on the different logical processors have different page
		tables to allow distinct stack and per-CPU storage pages transparently.
	\item \emph{IOBM} The file is a subject I/O bitmap. Specifies which I/O
		ports a subject is allowed to access.
	\item \emph{MSBM} The file is a subject MSR bitmap. Specifies which MSRs a
		subject is allowed to access.
	\item \emph{BIN} The file is a subject (raw) binary.
\end{itemize}

\begin{lstlisting}[
	caption=Example output of skpacker tool,
	label=lst:skpacker,
	frame=none,
	numbers=none]
         Packaging kernel image 'obj/kernel'
         0000000000200000 [PML4] kernel (0)
         0000000000204000 [PML4] kernel (1)
         0000000000208000 [PML4] kernel (2)
         000000000020c000 [PML4] kernel (3)
         0000000000210000 [PML4] tau0
         0000000000214000 [IOBM] tau0
         0000000000216000 [MSBM] tau0
         0000000000217000 [BIN ] tau0
         0000000000240000 [PML4] vt
         0000000000244000 [IOBM] vt
         0000000000246000 [MSBM] vt
         0000000000247000 [BIN ] vt
         0000000000270000 [PML4] crypter
         0000000000274000 [IOBM] crypter
         0000000000276000 [MSBM] crypter
         0000000000277000 [BIN ] crypter
         00000000002a0000 [PML4] sm
         00000000002a4000 [IOBM] sm
         00000000002a6000 [MSBM] sm
         00000000002a7000 [BIN ] sm
         00000000002d0000 [PML4] xv6 (EPT)
         00000000002d4000 [IOBM] xv6
         00000000002d6000 [MSBM] xv6
         00000000002d7000 [BIN ] xv6
\end{lstlisting}

Once the packaging step is complete, the resulting OS image can be booted by any
Multiboot \cite{multiboot} compliant bootloader.

\subsection{Emulation}
To ease kernel development, the Muen project makes heavy use of emulation using
the Bochs IA-32 emulator. Bochs has support for multiple processors, APIC
emulation and VMX extensions among others. This allows to run the Muen example
system described in the following section \ref{sec:example-system} as
illustrated by figure \ref{fig:bochs}.

\begin{figure}[h]
	\centering
	\includegraphics[width=\textwidth]{images/bochs}
	\caption{Bochs running the Muen example system}
	\label{fig:bochs}
\end{figure}

Bochs writes detailed logs during emulation and also provides a debugger which
allows to inspect the complete system state at any time. It has proven very
helpful when implementing new low-level processor features.

\section{Example system}\label{sec:example-system}
The Muen project contains an example system that makes use of all the
mechanisms described in the previous sections. Figure \ref{fig:example-system}
shows a schematic overview of the system.

\begin{figure}[h]
	\centering
	\begin{tikzpicture}[node distance=0.33cm]
	\node[redbox, text width=1.5cm, minimum width=2cm, minimum height=2cm] (vts) {VT Native};
	\node[redbox, text width=1.5cm, minimum width=2cm, minimum height=2cm, left=of vts] (cry) {Crypter Native};
	\node[redbox, text width=1.5cm, minimum width=2cm, minimum height=2cm, left=of cry] (smn) {Subject Monitor Native};
	\node[blackbox, text width=1.5cm, minimum width=2cm, minimum height=2cm, left=of smn] (xv6) {xv6 VM};
	\node[bluebox, minimum height=1cm, minimum width=9cm, text width=6cm] at (-3.5,-2.5) (mue) {Muen Separation Kernel};

	\draw[gray!80] (-8,-2.25) to (1,-2.25);
	\draw[gray!80] (-5.84,-2.25) to (-5.84,0);
	\draw[gray!80] (-3.50,-2.25) to (-3.50,0);
	\draw[gray!80] (-1.14,-2.25) to (-1.14,0);
\end{tikzpicture}

	\caption{Example system}
	\label{fig:example-system}
\end{figure}

The system is composed of the Muen kernel and four subjects, three of which are
trusted and one xv6 subject is untrusted. The trusted subjects run in the native
profile and are implemented in Ada/SPARK, the untrusted xv6 subject runs a teaching
OS written in C inside the VM profile.

The example system is meant to serve as demonstrator for a real-world use-case:
An untrusted operating system is separated by the Muen kernel and accesses
native, trusted services with minimal TCB. The trusted services might even be
formally verified.

The following section explains the different subjects composing the example
system, while section \ref{subsec:keyboard-handling} describes the keyboard
handling in detail to illustrate how the different mechanisms are tied
together.

\subsection{Subjects}

\subsubsection{VT}\label{subsubsec:subject-vt}
The VT subject manages virtual terminal consoles and owns the keyboard. The
system policy therefore assigns the hardware devices shown by listing
\ref{lst:hardware-devs} to the subject. This allows the VT subject to control
the VGA cursor and the contents of the VGA memory. The kernel also programs the
system's I/O APIC so that only the VT subject receives keyboard interrupts (IRQ
1).

\begin{lstlisting}[
	language=xml,
	label=lst:hardware-devs,
	caption=Subject device assignment]
<device name="keyboard" irq="1">
    <io_port start="0060" end="0060"/>
    <io_port start="0064" end="0064"/>
</device>

<device name="vga">
    <memory_layout>
        <memory_region physical_address="b8000" virtual_address="b8000" ...
    </memory_layout>
</device>

<device name="cursor">
    <io_port start="03d4" end="03d5"/>
</device>
\end{lstlisting}

The other subjects of the example system have no direct access to the real VGA
memory but write to a distinct page mapped to their (virtual) VGA memory
address (\emph{0xb8000}). The subject virtual terminal pages are also mapped
into the address space of the VT subject.

If the user hits the special keys F1 to F6, the VT subjects updates the VGA
memory with the contents of the active session slot's virtual terminal page.
All other keyboard scancodes are copied to a driver page shared with the
untrusted xv6 subject and an event is sent to inform it that keyboard data
awaits processing by its virtual keyboard driver.

\subsubsection{Crypter}
The crypter subject uses the libsparkcrypto \cite{libsparkcrypto} library to
provide trusted cryptographic services to clients. On startup, the subject
enters the halted state until it receives an interrupt event indicating a
pending request.

The interrupt event resumes the subject and data contained in the subject's
request page is copied for further processing. Currently, the crypter subject
creates a SHA-256 REF message digest over the received data and then copies the
hash to the service response page. An interrupt event is triggered to signal
service completion.

\subsubsection{xv6}\label{subsubsec:xv6}
Xv6\index{xv6} is a Version 6 Unix \cite{wiki:unix6} teaching operating system
developed at MIT \cite{xv6}. While being simple, it implements many key
concepts found in common operating systems, making it an ideal initial target
for the VM profile. Xv6 is written in ANSI C.

Minimal changes in the source code of xv6 were required to run it as VM subject
on the Muen kernel:
\begin{itemize}
	\item Disable MP support
	\item Ignore disallowed I/O operations (handled by the SM subject)
\end{itemize}

Since the xv6 subject has no direct access to the keyboard, a simple virtual
keyboard driver has been implemented.

\subsubsection{Subject Monitor}
The subject monitor (SM\index{SM}) subject is used to monitor the untrusted xv6
subject. It displays information about I/O operations and has complete access to
the architectural state of the xv6 subject. Currently, no emulation is needed to
run xv6. If an unexpected trap occurs, a state dump is output and the virtual
CPU (VCPU\index{VCPU}) is halted.

\subsection{Keyboard handling}\label{subsec:keyboard-handling}
This section describes how the keyboard is handled in the example system using
the mechanisms provided by the Muen kernel.

When pressing a key on the keyboard, the keyboard controller raises an interrupt
request (IRQ\index{IRQ}) with number 1 to signal new data. The system policy
applied by the Muen kernel on startup enforces that this interrupt is routed to
the appropriate processor running the VT subject. This is done using the IRQ
routing specification generated during policy compilation and depends on the
assignment of hardware devices to subjects. In this case, the keyboard is
assigned to subject 1, which is the VT subject described in section
\ref{subsubsec:subject-vt}. Listing \ref{lst:ex-irq-routing} shows the IRQ
routing table of the example system.

\begin{lstlisting}[
	language=Ada,
	label=lst:ex-irq-routing,
	caption=Example system IRQ routing table]
IRQ_Routing : constant IRQ_Routing_Array := IRQ_Routing_Array'(
  1 => IRQ_Route_Type'(
    CPU    => 0,
    IRQ    => 1,
    Vector => 33));
\end{lstlisting}

The routing table contains one entry which routes the keyboard IRQ 1 to the CPU
with APIC ID 0. Furthermore, the IRQ is remapped to the interrupt vector 33 so
that on each key press, CPU0 will be interrupted with an interrupt vector 33.

The Muen kernel running on CPU0 handles the received interrupt in its
\texttt{Handle\_Irq} procedure. It uses the vector routing table contained in
the SPARK-compliant interrupt specification compiled by the \texttt{skpolicy}
tool, see listing \ref{lst:ex-vector-routing}. The table instructs the kernel to
inject interrupt vector 33 into the subject with ID 1 by using the interrupt
injection mechanism described in section \ref{subsec:int-injection}.

\begin{lstlisting}[
	language=Ada,
	label=lst:ex-vector-routing,
	caption=Example system CPU0 vector routing table]
Vector_Routing : constant Vector_Routing_Array := Vector_Routing_Array'(
  33     => 1,
  others => Skp.Invalid_Subject);
\end{lstlisting}

The VT subject receives interrupt vector 33 in its interrupt handling routine
and, because it is allowed to access the respective I/O ports, reads in the
keyboard scancodes. If the received scancode is not related to the special keys
F1 to F6 and the currently visible subject is xv6, the scancode data is copied
into a memory page shared with the xv6 subject. The VT subject then triggers an
event to inform the xv6 subject about new data to process. This is done using
the Muen hypercall mechanism.

The kernel running the VT subject forwards the interrupt event to the respective
subject as defined by the policy. In this case, it is the xv6 subject.

The xv6 subject's interrupt handler code runs unmodified and detects that a
keyboard interrupt has occurred. The appropriate keyboard handling code is
called, which has been slightly modified to read keyboard scancodes from a
shared memory page instead of doing port I/O. The copied scancode data is then
used to drive the terminal console.

