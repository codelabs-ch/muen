\section{Example system}\label{sec:example-system}
The Muen project contains an example system that makes use of all the
mechanisms described in the previous sections. Figure \ref{fig:example-system}
shows a schematic overview of the system.

\begin{figure}[h]
	\centering
	\begin{tikzpicture}[node distance=0.33cm]
	\node[redbox, text width=1.5cm, minimum width=2cm, minimum height=2cm] (vts) {VT Native};
	\node[redbox, text width=1.5cm, minimum width=2cm, minimum height=2cm, left=of vts] (cry) {Crypter Native};
	\node[redbox, text width=1.5cm, minimum width=2cm, minimum height=2cm, left=of cry] (smn) {Subject Monitor Native};
	\node[blackbox, text width=1.5cm, minimum width=2cm, minimum height=2cm, left=of smn] (xv6) {xv6 VM};
	\node[bluebox, minimum height=1cm, minimum width=9cm, text width=6cm] at (-3.5,-2.5) (mue) {Muen Separation Kernel};

	\draw[gray!80] (-8,-2.25) to (1,-2.25);
	\draw[gray!80] (-5.84,-2.25) to (-5.84,0);
	\draw[gray!80] (-3.50,-2.25) to (-3.50,0);
	\draw[gray!80] (-1.14,-2.25) to (-1.14,0);
\end{tikzpicture}

	\caption{Example system}
	\label{fig:example-system}
\end{figure}

The system is composed of the Muen kernel and four subjects, three of which are
trusted and one xv6 subject is untrusted. The trusted subjects run in the native
profile and are implemented in Ada/SPARK, the untrusted xv6 subject runs a teaching
OS written in C inside the VM profile.

The example system is meant to serve as demonstrator for a real-world use-case:
An untrusted operating system is separated by the Muen kernel and accesses
native, trusted services with minimal TCB. The trusted services might even be
formally verified.

The following subsection explains the different subjects composing the example
system, while section \ref{subsec:use-cases} describes two use-cases in detail:
keyboard handling and the hasher service.

\subsection{Subjects}

\subsubsection{VT}
The VT subject manages virtual terminal consoles and owns the keyboard. The
system policy therefore assigns the hardware devices shown by listing
\ref{lst:hardware-devs} to the subject. This allows the VT subject to control
the VGA cursor and the contents of the VGA memory. The kernel also programs the
system's I/O APIC so that only the VT subject receives keyboard interrupts (IRQ
1).

\begin{lstlisting}[
	language=xml,
	label=lst:hardware-devs,
	caption=Subject device assignment]
<device name="keyboard" irq="1">
    <io_port start="0060" end="0060"/>
    <io_port start="0064" end="0064"/>
</device>

<device name="vga">
    <memory_layout>
        <memory_region physical_address="b8000" virtual_address="b8000" ...
    </memory_layout>
</device>

<device name="cursor">
    <io_port start="03d4" end="03d5"/>
</device>
\end{lstlisting}

The other subjects of the example system have no direct access to the real VGA
memory but write to a distinct page mapped to their (virtual) VGA memory
address (\emph{0xb8000}). The subject virtual terminal pages are also mapped
into the address space of the VT subject.

If the user hits the special keys F1 to F6, the VT subjects updates the VGA
memory with the contents of the active session slot's virtual terminal page.
All other keyboard scancodes are copied to a driver page shared with the
untrusted xv6 subject and an event is sent to inform it that keyboard data
awaits processing by its virtual keyboard driver.

\subsubsection{Crypter}
The crypter subject uses the libsparkcrypto \cite{libsparkcrypto} library to
provide trusted cryptographic services to clients. On startup, the subject
enters the halted state until it receives an interrupt event to signal that a
request is pending.

The interrupt event resumes the subject and data contained in the subject's
request page is copied for further processing. Currently, the crypter subject
creates a SHA-256 REF message digest over the reveived data and then copies the
hash to the service response page. An interrupt event is triggered to signal
service completion.

\subsubsection{xv6}
Xv6\index{xv6} is a Version 6 Unix \cite{wiki:unix6} teaching operating system
developed at MIT \cite{xv6}. While being simple, it implements many key
concepts found in common operating systems, making it an ideal initial target
for the VM profile. Xv6 is written in ANSI C.

Minimal changes in the source code of xv6 were required to run it as VM subject
on the Muen kernel:
\begin{itemize}
	\item Disable MP support
	\item Ignore disallowed I/O operations (handled by the SM subject)
\end{itemize}

Since the xv6 subject has no direct access to the keyboard, a simple virtual
keyboard driver has been implemented.

\subsubsection{Subject Monitor}
The subject monitor (SM\index{SM}) subject is used to monitor the untrusted xv6
subject. It displays information about I/O operations and has complete access to
the architectural state of the xv6 subject. Currently, no emulation is needed to
run xv6. If an unexpected trap occurs, a state dump is output and the virtual
CPU (VCPU\index{VCPU}) is halted.

\subsection{Use-cases}\label{subsec:use-cases}
\subsubsection{Keyboard handling}
\subsubsection{Hasher service}
