\subsection{Scheduling}\label{subsec:scheduling}
After system initialization is complete, the kernels running on the different
logical processors synchronize and start scheduling subjects. Which subject to
schedule on which CPU is defined by the scheduling plan. A policy writer defines
the system's scheduling regime in the XML policy file, see listing
\ref{lst:xml-scheduling-plan} for an example.

\begin{lstlisting}[
	language=xml,
	label=lst:xml-scheduling-plan,
	caption=System scheduling plan in XML]
<scheduling tick_rate="10000">
	<major_frame>
		<cpu>
			<minor_frame subject_id="1" ticks="40"/>
			<minor_frame subject_id="2" ticks="40"/>
		</cpu>
		<cpu>
			<minor_frame subject_id="3" ticks="80"/>
		</cpu>
	</major_frame>
	<major_frame>
		<cpu>
			<minor_frame subject_id="1" ticks="80"/>
		</cpu>
		<cpu>
			<minor_frame subject_id="4" ticks="80"/>
		</cpu>
	</major_frame>
</scheduling>
\end{lstlisting}

The example scheduling plan of listing \ref{lst:xml-scheduling-plan} contains
four subjects which are scheduled on two logical processors. The Muen policy
compiler explained in section \ref{subsec:policy-compilation} transforms the
scheduling plan in XML format to SPARK specifications which are directly
compiled into the kernel. The scheduling plan is therefore static at compilation
time and cannot be modified at runtime.

\begin{figure}[h]
	\centering
	\begin{tikzpicture}[minimum height=0.6cm]
	\node (sch) [bluebox]                {Scheduler};
	\node (knl) [bluebox, left=of sch]   {Kernel Main};
	\node (pln) [apribox, above=of sch]  {Scheduling Plan};
	\node (sub) [greenbox, right=of sch] {Subject};
	\node[gray, font=\scriptsize] at (0.8,2.3) {VMX root};
	\node[gray, font=\scriptsize] at (2.6,2.3) {VMX non-root};

	\draw[arrow] (knl) to (sch);
	\draw[arrow] (pln) to (sch);
	\draw[arrow] (sch) to[bend right=65] node[auto] {VM enter} (sub);
	\draw[arrow] (sub) to[bend right=65] node[auto] {VM exit}  (sch);
	\draw[thin, dotted, gray] (1.6,-1.5) to (1.6,2.5);
\end{tikzpicture}

	\caption{Kernel scheduler}
	\label{fig:kernel-scheduler}
\end{figure}

Each kernel stores the index of the active minor frame, which points into the
current scheduling major frame. Initially, this value is set to one
(\texttt{Minor\_Frame\_Range'First}), i.e. the first minor frame in the active
major frame.

The index designating the current major frame is global and therefore identical
for all kernels. This index is managed by the privileged $\tau$0 subject and
only read by the kernels.

The minor/major frame index tuple points into the scheduling plan. It points to
a minor frame containing the subject ID and timer ticks for the next subject to
schedule. The subject ID is used to load the state of the corresponding subject
from the state descriptor array into the processor. The initial state of a
subject is always initialized to null. The timer ticks are written into the
subject's VMCS region so that the subject is preempted automatically by the
processor when the alloted time slice is over. The kernel then calls the
\texttt{VMLAUNCH} or \texttt{VMRESUME} (if the subject has already been
launched) VT-x instructions to enter VMX non-root operation, which lets the
processor execute subject code. Figure \ref{fig:kernel-scheduler} illustrates
this process.

The subject state is saved on each VM exit. Depending on the scheduling plan and
the exit reason, the state of another subject is loaded by the kernel scheduler.
If the exit occurred because the VMX preemption timer fired, the scheduler is
aware that it must advance to the next minor frame in the current major frame.
This is done by incrementing the current minor frame counter. If the minor frame
index reaches the upper end of the allowed range
(\texttt{Minor\_Frame\_Range'Last}), it is reset to the first value of the
range.
