\section{Future work}

The following list gives a somewhat ordered overview of possible future
enhancements to the Muen separation kernel:

\begin{itemize}
	\item Covert/Side-Channel analysis
	\item Linux virtualization
	\item Hardware passthrough/PCIe virtualization
	\item Dynamic resource management
	\item Multicore subjects
	\item Performance optimization
	\item Power Management
	\item Formal verification
	\item Fully virtualized subjects
\end{itemize}

The following sections describe some of the more interesting items.

\subsection{Covert/Side-channel analysis}

\subsection{Linux subject}

\subsection{PCIe virtualization}

\subsection{Dynamic resource management}
Currently, resource management is static and there is little flexibility.
Switching between pre-defined scheduling plans is the only property
configurable at runtime. While this allows to tightly control and validate a
system via the policy, all assigned resources are indispensable even though some
of them may not be in use during a particular period.

For example, a system consisting of multiple different VM subjects with their
respective operating systems need multiple gigabytes of memory when running.
In a static system configuration, all the assigned memory is committed and
reserved even though only one of the VMs might be in use. A system that would
allow dynamic launching and stopping of subjects, only the resources of the
currently active subjects would be in use.

This can be achieved by introducing a trusted subject named $\tau$0. It is has
the same trust-level as the kernel and forms part of the TCB. Via an appropriate
interface it can update parts of the policy and instruct the kernel to perform
the necessary actions to enforce the changes.

\subsection{Formal verification}
By implementing the kernel in SPARK and proving the absence of runtime errors,
we have shown that the kernel is free from exceptions. While these proofs
provide some evidence to the correctness claim of the implementation, the
application of these particular formal methods do not provide any assurances
beyond the error free execution of the kernel. Proving functional properties
such as the correspondence of the scheduler to a given formal specification is
necessary to further raise the confidence in systems based on the Muen kernel.
