\section{Requirements}

The following properties specify the requirements of the separation kernel and
the system's TCB:
\begin{enumerate}
	\item All resources must be protected from unauthorized access. This
		includes devices and memory that has not been assigned to any subject.
	\item System resources internal to the kernel, that are not exported to
		subjects, must not be accessible.
	\item Isolated subjects that do not share any resources must be completely
		separated and not be able to exchange any information.
	\item Information flows between subjects must only be present if explicitly
		specified in the system policy.
	\item It must be possible to specify directed information flows, where data
		can be transfered from source to destination but not in the reverse
		direction.
	\item A subject must be able to only access its assigned resources. Access
		to other resources must be prohibited.
	\item Assignment of subject resources must be static. A malicious subject
		must not be able to gain access to additional resources or even consume
		all system resources.
	\item All security critical operations must be NEAT: non-bypassable,
		evaluatable, always-invoked and tamperproof.
	\item 64-bit programs must be supported to run as native subjects.
	\item The kernel must support the Intel IA-32e/64-bit architecture and
		system memory larger than 4GB.
	\item The kernel must allow the implementation of small and simple
		components and not impose unnecessary restrictions that would increase
		subject complexity.
	\item Subjects must be able to use hardware devices and process device
		generated interrupts in particular.
	\item A mechanism for inter-subject notifications must exist.
	\item The kernel must be policy-free and only provide mechanisms to enforce
		a given policy.
\end{enumerate}
