\chapter{Background}
\section{SPARK}
SPARK is a formally-define high-level programming language designed for writing
high integrity systems. It is based on Ada, which is itself a programming
language with a strong focus on security and safety.

The SPARK language is a subset of Ada with additional features inserted as
annotations by means of Ada comments. Since compilers ignore comments and SPARK
is a true subset of Ada, any correct SPARK program is a correct Ada programm and
can be compiled using existing Ada compilers, such as GNAT, which is part of the
GNU compiler collection (GCC) TODO:ref. However, since annotations are an
integral part of SPARK, it would be misleading to simply consider SPARK a
constrained version of Ada. SPARK should be viewed as a programming language in
its own right.

The annotations are processed by SPARK tools. These tools perform static
analysis of the source code. The annotations allow the tools to do data and
information flow analysis as well as proof the absence of runtime errors. This
means that SPARK tools allow to formally verify, that a given program is free
of errors such as division by zero, out-of-bounds array access etc. The
following program properties can be formally proven using SPARK:

\begin{itemize}
	\item TODO
\end{itemize}

On top of these properties, the usage of pre-/post-conditions and assertions
allow to prove additional functional properties. A proof of (partial
\footnote{Termination cannot be shown}) correctness of SPARK programs is
achievable. This allows to formally show the correspondence of an implementation
with its formal specification.

It is also interesting to note, that has support for tasking in the form a
profile called RavenSPARK TODO:Ref.

SPARK is a mature technology and has garnered quite some interest since it has
been used successfully in several industrial projects TODO:ref. It is primarily
employed in the field of avionics, space, medical systems and in the defense
industry.

\subsection{Design rationale}
The main driving factors behind the design of SPARK are shortly described here:

\begin{description}
	\item[Logical soundness] \hfill \\
		The language must not contain any ambiguities and defined specified.
	\item[Complexity of formal language definition] \hfill \\
		The language must be simple to specify formally.
	\item[Expressive power] \hfill \\
		The language must have rich enough to implement real systems.
	\item[Security] \hfill \\
		All language rules must be statically checkable with reasonable effort
		(within polynomial time).
	\item[Verifiability] \hfill \\
		Program verification must be tractable for industrial scale projects.
	\item[Bounded space and time requirements] \hfill \\
		Resource requirements must be determined statically.
\end{description}

\subsection{Example}
The following listing illustrates how annotations are used to specify the
contract of a subprogram.

\begin{lstlisting}[language=Ada]
type Color_Type is (Red, Green, Blue);

procedure Exchange (X, Y: in out Color_Type);
--# derives X from Y &
--#         Y from X;
--# post X = Y~ and Y = X~;
\end{lstlisting}

An in-depth discussion of the SPARK programming language can be found in
\cite{BarnesSPARK}.

\subsection{SPARK 2014}

\section{Intel x86 Architecture}
\section{Virtualization}
\section{Motivation}
\section{Goals}
\section{Related work}
